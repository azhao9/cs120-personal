\documentclass{article}
\usepackage[sexy, hdr, fancy]{evan}
\setlength{\droptitle}{-4em}

\lhead{Assignment 3}
\rhead{Intermediate Programming}
\lfoot{}
\cfoot{\thepage}

\lstset{language=C}
\lstdefinestyle{nonumbers}
{numbers=none}

\begin{document}
\title{Assignment 3}
\maketitle
\thispagestyle{fancy}

\section*{Code Puzzles}
\begin{itemize}
	\item[1.]
	\begin{lstlisting}[style=nonumbers]
	double x = 3.1415926535;
	for (int i = 0; i < 8; i++) {
		int y = x;
		printf("%d", y);
		x -= y;
		x *= 10.0;
	}
	\end{lstlisting}
	\begin{answer*}
		This prints out 
		\begin{lstlisting}[style=nonumbers]
		31415926
		\end{lstlisting}
	\end{answer*}
		
	\item[2.]
	\begin{lstlisting}[style=nonumbers]
	int i = 0;
	int j = 1;
	int k = -1;
	bool a = true;
	bool b = false;

	if (b = true)
		printf("1");
	if (j || (a = false))
		printf("2");

	if (i || a)
		printf("3");
	else
		printf("4");
		printf("5");

	if (j == k) {
		printf("6");
	} else if (b && k) {
		printf("7");
	} else {
		printf("8");
	}

	for (i = 0; i < 5; i++);
	{
		j++;
	}

	if (i > j) {
		printf("9");
	}
	\end{lstlisting}
	\begin{answer*}
		This prints out
		\begin{lstlisting}[style=nonumbers]
		123579
		\end{lstlisting}
	\end{answer*}
	
	\item[3.]
	\begin{lstlisting}[style=nonumbers]
	int y = 7;
	printf("%d\n", y);
	for (int x = 0; x < 10; x++) {
		int y = x * 2;
		printf("%d ", y);
	}
	printf("\n%d\n", y);
	\end{lstlisting}
	\begin{answer*}
		This prints out
		\begin{lstlisting}[style=nonumbers]
		7
		0 2 4 6 8 10 12 14 16 18
		7
		\end{lstlisting}
	\end{answer*}
		
	\item[4.] 
	\begin{lstlisting}[style=nonumbers]
	char str[10] = "strange";
	int len = strlen(str);
	printf("%d\n", len);

	for (int i = 2; i < len - 2; i++) {
		printf("%c", str[len - i]);
	}
	\end{lstlisting}
	\begin{answer*}
		This prints out
		\begin{lstlisting}[style=nonumbers]
		7
		gna
		\end{lstlisting}
	\end{answer*}
	
	\item[5.] 
	\begin{lstlisting}[style=nonumbers]
	int thing1(int i) {
		static int j = 5;
		j += i;
		i = j * 2;
		return i;
	}

	int thing2(int x, int y) {
		int z = thing1(x) + thing1(y) + x + y;
		return z;
	}

	int main() {
		int a = 2;
		int b = 5;
		printf("a=%d, b=%d, thing2(a, b)=%d", a, b, thing2(a, b));
		return 0;
	}
	\end{lstlisting}
	\begin{answer*}
		This prints out
		\begin{lstlisting}[style=nonumbers]
		a=2, b=5, thing2(a, b)=45
		\end{lstlisting}
	\end{answer*}
		
	\item[6.] 
	\begin{lstlisting}[style=nonumbers]
	int i;
	for (i = 0; i < 100; i++) {
		while (i % 10) {
			i++;
		}
		printf("%d ", i);
	}
	\end{lstlisting}
	\begin{answer*}
		This prints out 
		\begin{lstlisting}[style=nonumbers]
		0 10 20 30 40 50 60 70 80 90 100
		\end{lstlisting}
		
	\end{answer*}

\end{itemize}

\section*{Code Reading}
\begin{itemize}
	\item[7.] 
	\begin{lstlisting}[style=nonumbers]
	/* takes an array and an integer specifying the length of the array */
	double arrayFunc(int array[], int length) {
		if (length < 1) {
			return 0;
		}

		double s = 0;
		for (int i = 0; i < length; i++) {
			s += array[i];
		}
		return s/length;
	}
	\end{lstlisting}
	\begin{answer*}
		This returns the double precision average of all elements in the array.
		
	\end{answer*}
	
	\item[8.]
	\begin{lstlisting}[style=nonumbers]
	for (int n = 2; n < 100; n++) {
		int p = 1;
		for (int d = 2; p && (d < n); d++) {
			if (n % d == 0) {
				p = 0;
			}
		}
		if (p) {
			printf("%d ", n);
		}
	}
	\end{lstlisting}
	\begin{answer*}
		This prints all prime numbers less than 100 and beginning with 2.

	\end{answer*}
	
\end{itemize}

\section*{Code Writing}
\begin{itemize}
	\item[9.] Write a function that takes a character array (assume it is a properly formatted C-style string), and prints the complete string to stdout backwards. Be sure to print all (and only) the characters of the string.

			\begin{lstlisting}[style=nonumbers]
			void backstr(char string[]) {
				int len = strlen(string);
				
				for (int i = len - 1; i >= 0; i--) {
					printf("%c", string[i]);
				}
				
				printf("\n");
			}
			\end{lstlisting}	

			\newpage
	\item[10.] Write a loop that allows a user to enter as many integers as she wants, and then prints out the sum of those numbers (no arrays should be required). Use scanf() for input (remember that scanf() returns the number of things it successfully read, so you can just read numbers until scanf() fails to get an integer).

		\begin{lstlisting}[style=nonumbers]
			int main() {

				int sum = 0;
				int cur;

				printf("Enter integers:\n");
				int scan = scanf("%d", &cur);

				while (scan > 0) {
					sum += cur;

					scan = scanf("%d", &cur);
				}

				printf("\nSum: %d\n", sum);

				return 0;
			}
		\end{lstlisting}

\end{itemize}

\section*{Explanation}
\begin{itemize}
	\item[11.] Explain why global-scope variables are considered to be poor practice.
		\begin{answer*}
			They make debugging difficult because if multiple functions operate on a global-scope variable, it can be hard to find which one is causing the problem.
			
		\end{answer*}

	\item[12.] Explain why you need to specify .c files but not .h files on the command line when you call gcc.
		\begin{answer*}
			.h files are specified at the header of each .c file, so they don't need to be specified again.
			
		\end{answer*}

	\item[13.] Explain why the C language is commonly used for writing operating systems, and why it is used less often for small, day-to-day programming tasks.
		\begin{answer*}
			C can directly work on the hardware, which makes it extremely fast - good for operating systems. It isn't used as often for small tasks because it doesn't have much built in functionality and direct communication with the hardware can be unsafe.
			
		\end{answer*}
\end{itemize}

\end{document}
